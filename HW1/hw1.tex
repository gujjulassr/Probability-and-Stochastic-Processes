

\let\negmedspace\undefined
\let\negthickspace\undefined
\documentclass[journal,12pt,twocolumn]{IEEEtran}
%\documentclass[conference]{IEEEtran}
%\IEEEoverridecommandlockouts
% The preceding line is only needed to identify funding in the first footnote. If that is unneeded, please comment it out.
\usepackage{cite}
\usepackage{amsmath,amssymb,amsfonts,amsthm}
\usepackage{amsmath}
\usepackage{algorithmic}
\usepackage{graphicx}
\usepackage{textcomp}
\usepackage{xcolor}
\usepackage{txfonts}
\usepackage{listings}
\usepackage{enumitem}
\usepackage{mathtools}
\usepackage{gensymb}
\usepackage[breaklinks=true]{hyperref}
\usepackage{tkz-euclide} % loads  TikZ and tkz-base
\usepackage{listings}

\graphicspath{ {./images/} }

\DeclareMathOperator*{\Res}{Res}
\renewcommand\thesection{\arabic{section}}
\renewcommand\thesubsection{\thesection.\arabic{subsection}}
\renewcommand\thesubsubsection{\thesubsection.\arabic{subsubsection}}

\renewcommand\thesectiondis{\arabic{section}}
\renewcommand\thesubsectiondis{\thesectiondis.\arabic{subsection}}
\renewcommand\thesubsubsectiondis{\thesubsectiondis.\arabic{subsubsection}}

% correct bad hyphenation here
\hyphenation{op-tical net-works semi-conduc-tor}
\def\inputGnumericTable{}                                 %%

\lstset{
%language=C,
frame=single, 
breaklines=true,
columns=fullflexible
}


\begin{document}
%


\newtheorem{theorem}{Theorem}[section]
\newtheorem{problem}{Problem}
\newtheorem{proposition}{Proposition}[section]
\newtheorem{lemma}{Lemma}[section]
\newtheorem{corollary}[theorem]{Corollary}
\newtheorem{example}{Example}[section]
\newtheorem{definition}[problem]{Definition}

\newcommand{\BEQA}{\begin{eqnarray}}
\newcommand{\EEQA}{\end{eqnarray}}
\newcommand{\define}{\stackrel{\triangle}{=}}
\bibliographystyle{IEEEtran}

\providecommand{\mbf}{\mathbf}
\providecommand{\pr}[1]{\ensuremath{\Pr\left(#1\right)}}
\providecommand{\qfunc}[1]{\ensuremath{Q\left(#1\right)}}
\providecommand{\sbrak}[1]{\ensuremath{{}\left[#1\right]}}
\providecommand{\lsbrak}[1]{\ensuremath{{}\left[#1\right.}}
\providecommand{\rsbrak}[1]{\ensuremath{{}\left.#1\right]}}
\providecommand{\brak}[1]{\ensuremath{\left(#1\right)}}
\providecommand{\lbrak}[1]{\ensuremath{\left(#1\right.}}
\providecommand{\rbrak}[1]{\ensuremath{\left.#1\right)}}
\providecommand{\cbrak}[1]{\ensuremath{\left\{#1\right\}}}
\providecommand{\lcbrak}[1]{\ensuremath{\left\{#1\right.}}
\providecommand{\rcbrak}[1]{\ensuremath{\left.#1\right\}}}
\theoremstyle{remark}
\newtheorem{rem}{Remark}
\newcommand{\sgn}{\mathop{\mathrm{sgn}}}
\providecommand{\abs}[1]{\left\vert#1\right\vert}
\providecommand{\res}[1]{\Res\displaylimits_{#1}} 
\providecommand{\norm}[1]{\left\lVert#1\right\rVert}

\providecommand{\mtx}[1]{\mathbf{#1}}
\providecommand{\mean}[1]{E\left[ #1 \right]}
\providecommand{\fourier}{\overset{\mathcal{F}}{ \rightleftharpoons}}

\providecommand{\system}{\overset{\mathcal{H}}{ \longleftrightarrow}}
	
\newcommand{\solution}{\noindent \textbf{Solution: }}
\newcommand{\cosec}{\,\text{cosec}\,}
\providecommand{\dec}[2]{\ensuremath{\overset{#1}{\underset{#2}{\gtrless}}}}
\newcommand{\myvec}[1]{\ensuremath{\begin{pmatrix}#1\end{pmatrix}}}
\newcommand{\mydet}[1]{\ensuremath{\begin{vmatrix}#1\end{vmatrix}}}

\let\vec\mathbf

\vspace{3cm}
\title{

Probability Assignment -I

}
\author{ Gujjula Samarasimha Reddy(AI23MTECH02001)$^{}$
}	

\maketitle
\newpage

\bigskip
\renewcommand{\thefigure}{\theenumi}
\renewcommand{\thetable}{\theenumi}

\textbf{Question:}\\
Two dice are thrown at the same time. Find the probability of getting
\begin{enumerate}
     \item same number on both dice.
     \item different numbers on both dice.
\end{enumerate}
 \textbf{Solution:}\\ \\ 
 \underline {(1)} \\ \\
 Let S represents the total sample space when two dice are rolled
 \\Then \\ \\
 $ S \in    \begin{Bmatrix}
             (1,1)& (1,2)& (1,3)& (1,4)& (1,5)& (1,6)\\
             (2,1)& (2,2)& (2,3)& (2,4)& (2,5)& (2,6)\\
             (3,1)& (3,2)& (3,3)& (3,4)& (3,5)& (3,6)\\
             (4,1)& (4,2)& (4,3)& (4,4)& (4,5)& (4,6)\\
             (5,1)& (5,2)& (5,3)& (5,4)& (5,5)& (5,6)\\
             (6,1)& (6,2)& (6,3)& (6,4)& (6,5)& (6,6)
    \end{Bmatrix}$ \\ \\ \\





\noindent There are total 36 possible outcomes when two dice are rolled and altogether represents the samplespace
mages\\

\noindent Let A be the event that represents getting  same number on the both the dice \\
     $A \in \{(1,1),(2,2),(3,3),(4,4),(5,5),(6,6)\}$ \\

     \noindent There are total 6 possible outcomes that all together represents the event A\\
     
    \begin{center}
    \includegraphics{dice}
    \end{center}

    \begin{align*}
        \pr{A}=\frac{6}{36}\\
        \pr{A}=\frac{1}{6}
   \end{align*}


   \underline {(2)} \\ \\

   \noindent From the above analysis we got to know that the propbability of getting same number as 
   \begin{align*}
    \pr{A}=\frac{1}{6}
\end{align*}

\noindent Let define {B} i\noindent s an event of getting different number on both the dice when they are rolled\\
Events {A} and {B} are mutually exclusive 

\noindent As we defined only two events on whole sample space and they are mutually exclusive  $ \colon $ 

\begin{align*}
 {A \cap B = \emptyset} \\
 \pr{A \cap B}={0}\\
 \therefore \pr{A \cup B}=\pr{A}+\pr{B}\\
 \pr{A} + \pr{B} = {1} \\
 \pr{B} = {1}-\pr{A} \\
 \pr{B}={1}-\frac{1}{6}\\
 \pr{B}=\frac{5}{6}
\end{align*}
\end{document}